%---------------------------------------------------------------------
%	 个人信息
%---------------------------------------------------------------------
\newcommand{\MYTITLE}{污秽之世,美丽之笼}  %论文标题
\newcommand{\MYID}{2018300000}  %学号
\newcommand{\MYNAME}{藤原妹红}  %姓名
\newcommand{\MYSCHOOL}{金融学院}  %学院
\newcommand{\MYMAJOR}{博丽神社}  %专业
\newcommand{\MYADVISOR}{上白泽慧音}  %指导教师
\newcommand{\MYDATE}{2024年4月10日}  %日期

%---------------------------------------------------------------------
% 文档类型
%---------------------------------------------------------------------
\documentclass[a4paper, 12bp, UTF8]{report}
%---------------------------------------------------------------------
% 导入宏包
%---------------------------------------------------------------------
\usepackage{geometry} % 改变页面尺寸
\usepackage[fontset=none]{ctex} % 中文支持
\usepackage{anyfontsize} % 支持任意字号
\usepackage{comment} % 添加注释环境
\usepackage{setspace} % 设置行间距
\usepackage{microtype} % 优化字距
\usepackage{fancyhdr} % 自定义页眉页脚
\usepackage[ruled]{manyfoot} % 多个脚注
\usepackage{graphicx} % 插入图像
\usepackage{wrapfig} % 文字环绕图像
\usepackage{subcaption} % 替代过时的 subfigure,支持子图与子标题
\usepackage{float} % 浮动体控制
\usepackage{enumitem} % 列表设置
\usepackage{verbatimbox} % 提供更多的 verbatim 环境
\usepackage{tabularx} % 增强表格功能
\usepackage{titlesec} % 修改章节标题样式
\usepackage{titletoc} % 修改目录样式
\usepackage[titletoc]{appendix} % 附录设置
\usepackage{listings, color, xcolor} % 插入代码
\usepackage{algpseudocode} % 代替过时的 algorithmic,用于编写伪代码
\usepackage{algorithm} % 插入算法伪代码环境
\usepackage{scrextend} % KOMA-Script 扩展
\usepackage[perpage, bottom]{footmisc} % 脚注相关设置
\usepackage{pifont} % 提供圆圈数字等特殊字符
\usepackage[T1]{fontenc} % 字体编码为 T1
\usepackage[
    backend=biber, % 编译后端
    citestyle=gb7714-2015ay, % 文献引用样式
    bibstyle=gb7714-2015, % 参考文献样式
    % backref=true, % 文献引用页码
    defernumbers=true, % 参考文献编号延迟
    gbnamefmt=lowercase, % 姓名格式
    doi=false, %显示doi与否
    url=false, %显示url与否
    % sorting=gb7714-2015, %以姓氏编号
    maxcitenames=2, %最多显示两个作者
    mincitenames=1, %最少显示一个作者
    % sortcites=false, %不排序
    sortlocale=zh__pinyin, %按拼音排序
    gbalign=left, %对齐方式,同时显示序号
    gbcitelabel=quanjiao %引用全角括号及标点
]{biblatex} % 参考文献管理
\usepackage{amsmath, amssymb, amsthm, amsfonts, mathtools} % 数学环境
\usepackage{tikz} % tikz绘图
\usetikzlibrary{arrows.meta, positioning, shapes.geometric} % tikz绘图库
\usepackage{lipsum} % 生成随机文本
\usepackage{tabularx, longtable, booktabs} % 更优美的表格
\usepackage{verbatim} % 更好的 verbatim 环境
\usepackage{cases} % 公式组环境
\usepackage{multirow} % 在表格中跨多行合并单元格
\usepackage{zhnumber} % 章节编号为中文
\usepackage{hyperref} % 超链接支持
\usepackage[labelsep=quad, singlelinecheck=false]{caption} % 设置图表标题为五号
\usepackage{chngcntr} % 允许改变计数器的前缀
%---------------------------------------------------------------------
%  页边距设置
%---------------------------------------------------------------------
\geometry{a4paper, left=2.5cm, right=2.5cm, top=2.54cm, bottom=2.54cm, xetex}
%---------------------------------------------------------------------
%  脚注设置
%---------------------------------------------------------------------
% 设置带圈数字
\newcommand*\dingctr[1]{\protect\ding{\number\numexpr\value{#1}+171\relax}}
\renewcommand*\thefootnote{\dingctr{footnote}}
% 脚注字体设置为small
\renewcommand{\footnotesize}{\small}
\deffootnote[1em]{1em}{1em}{\small\thefootnotemark\space}
% 脚注的横线
\renewcommand{\footnoterule}{
    \kern -3bp
    \hrule width 0.25\paperwidth height 1bp
    \kern 2bp
}
\setlength{\footnotesep}{12bp}
%---------------------------------------------------------------------
%  参考文献设置
%---------------------------------------------------------------------
\renewcommand*{\bibfont}{\normalsize}
\addbibresource{references/reference.bib}
% 引用文献设置为上标
%\begin{comment}
%        \makeatletter
%        \def\@cite#1#2{\textsuperscript{[{#1\if@tempswa , #2\fi}]}}
%        \makeatother
%\end{comment}
% 参考文献间距
\setlength{\bibitemsep}{0bp}
\setlength{\bibnamesep}{0bp}
\setlength{\bibitemindent}{0bp} % bibitemindent表示一条文献中第一行相对后面各行的缩进
\setlength{\bibhang}{0bp} % 著者-出版年制中 bibhang 表示的各行起始位置到页边的距离
%---------------------------------------------------------------------
%  附录代码设置
%---------------------------------------------------------------------
% Python highlights color settings
\definecolor{pBasic}{RGB}{248, 248, 242}      % 默认
\definecolor{pKeyword}{RGB}{228,0,128}        % 洋红色
\definecolor{pString}{RGB}{148,0,209}         % 紫色
\definecolor{pComment}{RGB}{117,113,94}       % 灰色
\definecolor{pIdentifier}{RGB}{166, 226, 46}  % 绿色
\definecolor{pBackground}{RGB}{245,245,245}   % 浅灰色
\definecolor{pNumber}{RGB}{128,128,128}       % 灰色
\lstdefinestyle{python}{
    language=python,               % 语言
    xleftmargin=25bp,
    xrightmargin=15bp,
    frame=tlbr,framesep=4bp,framerule=0bp, % 边框设置
    % frame=shadowbox,rulesepcolor=\color{red!20!green!20!blue!20}, % 边框设置
    basicstyle=\singlespacing \small \fontspec{Consolas},          % 代码字体、大小、行间距
    keywordstyle=\color{pKeyword},       % 关键字颜色
    stringstyle=\color{pString},         % 字符串颜色
    commentstyle=\color{pComment},       % 注释颜色
    % backgroundcolor=\color{pBackground}, % 背景颜色
    emph={format_string,eff_ana_bf,permute,eff_ana_btr}, % 自定义函数
    emphstyle=\color{pIdentifier},       % 自定义函数颜色
    showspaces=false,                    % 显示空格
    showstringspaces=false,              % 用下划线显示空格
    showtabs=false,                      % 显示tab
    tabsize=4,                           % 设置默认缩进空格数
    captionpos=t,                        % 设置caption在顶部
    breaklines=true,                     % 自动换行
    numberstyle=\small \color{pNumber},
    numbers=left,                        % 行号位置
    stepnumber=1,                        % 行号计数间隔
    %numbersep=5pt,                      % 行号与代码间距
}
%---------------------------------------------------------------------
% 超链接设置
%---------------------------------------------------------------------
\hypersetup{colorlinks,linkcolor=black,anchorcolor=black,citecolor=black, pdfstartview=FitH,bookmarksnumbered=true,bookmarksopen=true,} % 设置引用样式
\XeTeXlinebreaklocale "zh"
% \XeTeXlinebreakskip = 0bp plus 1bp minus 0.1bp % 微调 XeTeX 的行间距
%---------------------------------------------------------------------
% 列表格式设置
%---------------------------------------------------------------------
\setlist[enumerate]{left=2em, labelindent=2em, label=(\arabic*), itemsep=0bp, topsep=0bp, partopsep=0bp, parsep=\parskip}
\setlist[itemize]{left=2em, labelindent=2em, itemsep=0bp, topsep=0bp, partopsep=0bp, parsep=\parskip}
%---------------------------------------------------------------------
% 图表名设置
%---------------------------------------------------------------------
\captionsetup[table]{name={表}, justification=raggedright, position=above, aboveskip=0bp, belowskip=17bp}
\captionsetup[figure]{name={图}, justification=centering, position=below, aboveskip=0bp, belowskip=4bp}
% 表格行间距设置
\renewcommand\arraystretch{0.9}
%---------------------------------------------------------------------
% 图表顺序标号,不分章节
%---------------------------------------------------------------------
\counterwithout{table}{chapter}
\counterwithout{table}{section}
\counterwithout{figure}{chapter}
\counterwithout{figure}{section}
\counterwithout{equation}{chapter}
\counterwithout{equation}{section}
\titleclass{\chapter}{straight}%禁止chapter换页
%---------------------------------------------------------------------
% 在目录中添加不带编号的章节
%---------------------------------------------------------------------
\makeatletter
\newcommand\specialchapter{\setcounter{secnumdepth}{-2}}
\makeatother
%---------------------------------------------------------------------
% 页眉页脚设置
%---------------------------------------------------------------------
\pagestyle{fancy}
% \fancyhead[C]{\small\MYTITLE}
\lhead{}
\rhead{}
% 页眉去除横线
\renewcommand{\headrulewidth}{0bp}
% \cfoot{\thepage \\ \textcolor{red}{请注意格式问题可能会导致拒绝答辩。以任何形式采用该模板意味着您已承认:使用该模板而引发的一切负面或正面后果与任何你以外的人都没有任何关系。}}
% \setlength{\headheight}{18bp}
%---------------------------------------------------------------------
% 调整字体
%---------------------------------------------------------------------
\setmainfont{Times New Roman}
\usepackage{fontspec}
\newfontfamily\eheiti[AutoFakeBold=false]{SimHei}
\setCJKmainfont[AutoFakeBold=2.17, AutoFakeSlant]{SimSun}
\setCJKsansfont[AutoFakeBold=2.17, AutoFakeSlant]{SimHei}
\setCJKmonofont{FangSong}
\newCJKfontfamily\songti[AutoFakeBold=2.17, AutoFakeSlant]{SimSun}
\newCJKfontfamily\heiti[AutoFakeBold=2.17, AutoFakeSlant]{SimHei}
\newCJKfontfamily\kaishu{KaiTi}
\newCJKfontfamily\fangsong{FangSong}
\makeatletter
% 重定义\normalsize(小四)来设置字体为12bp和行间距为22bp
\renewcommand\normalsize{%
    \@setfontsize\normalsize{12bp}{17bp}%
    \abovedisplayskip 17\p@ \@plus0\p@ \@minus0\p@
    \abovedisplayshortskip \z@ \@plus0\p@
    \belowdisplayshortskip 17\p@ \@plus0\p@ \@minus0\p@
    \belowdisplayskip \abovedisplayskip
    \let\@listi\@listI}
% 重定义\small(五号)来设置字体为10.5bp和行间距为22bp
\renewcommand\small{%
    \@setfontsize\small{10.5bp}{17bp}%
    \abovedisplayskip 17\p@ \@plus0\p@ \@minus0\p@
    \abovedisplayshortskip \z@ \@plus0\p@
    \belowdisplayshortskip 17\p@ \@plus0\p@ \@minus0\p@
    \belowdisplayskip \abovedisplayskip
    \let\@listi\@listI}
% 重定义\large(四号)来设置字体为14bp和行间距为22bp
\renewcommand\large{%
    \@setfontsize\large{14bp}{17bp}%
    \abovedisplayskip 17\p@ \@plus0\p@ \@minus0\p@
    \abovedisplayshortskip \z@ \@plus0\p@
    \belowdisplayshortskip 17\p@ \@plus0\p@ \@minus0\p@
    \belowdisplayskip \abovedisplayskip
    \let\@listi\@listI}
% 重定义\Large(小三)来设置字体为15bp和行间距为22bp
\renewcommand\Large{%
    \@setfontsize\Large{15bp}{17bp}%
    \abovedisplayskip 17\p@ \@plus0\p@ \@minus0\p@
    \abovedisplayshortskip \z@ \@plus0\p@
    \belowdisplayshortskip 17\p@ \@plus0\p@ \@minus0\p@
    \belowdisplayskip \abovedisplayskip
    \let\@listi\@listI}
% 重定义\LARGE(三号)来设置字体为16bp和行间距为22bp
\renewcommand\LARGE{%
    \@setfontsize\LARGE{16bp}{17bp}%
    \abovedisplayskip 17\p@ \@plus0\p@ \@minus0\p@
    \abovedisplayshortskip \z@ \@plus0\p@
    \belowdisplayshortskip 17\p@ \@plus0\p@ \@minus0\p@
    \belowdisplayskip \abovedisplayskip
    \let\@listi\@listI}
\makeatother
%---------------------------------------------------------------------
% 标题格式设置
%---------------------------------------------------------------------
\setcounter{secnumdepth}{3}
\renewcommand\thechapter{\zhnum{chapter}、}
\renewcommand\thesection{(\zhnum{section})}
\renewcommand\thesubsection{\arabic{subsection}.$\ $}
\renewcommand\thesubsubsection{(\arabic{subsubsection})}
\renewcommand {\thetable} {\arabic{table}}
\renewcommand {\thefigure} {\arabic{figure}}

\titleformat{\chapter}{\centering\large\songti\bfseries}{\thechapter}{0em}{}
\titleformat{\section}{\normalsize\songti\bfseries}{\hspace{2em}\thesection}{0em}{}
\titleformat{\subsection}{\normalsize\songti\mdseries}{\hspace{2em}\thesubsection}{0em}{}
\titleformat{\subsubsection}[runin]{\normalsize\songti\mdseries}{\hspace{2em}\thesubsubsection}{0em}{}[:\qquad]

\titlespacing{\chapter}{0bp}{22bp}{22bp} % 一级标题前后隔一行小四
\titlespacing{\section}{0bp}{0bp}{0bp}
\titlespacing{\subsection}{0bp}{0bp}{0bp}
\titlespacing{\subsubsection}{0bp}{0bp}{0bp}

\newtheorem{theorem}{定理}
\newtheorem{definition}{定义}
\newtheorem{corollary}{推论}
\newtheorem{example}{例}
%---------------------------------------------------------------------
% 摘要设置
%---------------------------------------------------------------------
\newcommand{\enabstractname}{ABSTRACT}
\newcommand{\cnabstractname}{内\hspace{0.5em}容\hspace{0.5em}摘\hspace{0.5em}要}
\newenvironment{cnabstract}{%
    \begin{center}
        \songti \bfseries \LARGE \cnabstractname \vspace{12bp}
    \end{center}
    \setlength{\parindent}{2em}
}{%
    \par \vspace{12bp}
}

\newenvironment{enabstract}{%
    \begin{center}
        \selectfont \bfseries \LARGE \enabstractname \vspace{10bp}
    \end{center}
    \setlength{\parindent}{2em}
}{%
    \par % \vspace{22bp}
}
%---------------------------------------------------------------------
% 目录页设置
%---------------------------------------------------------------------
\setcounter{tocdepth}{1}
\renewcommand{\contentsname}{\LARGE\bfseries\centering{目\hspace{1em}录}}
\titlecontents{chapter}[0em]{\large\songti\bfseries}{\thecontentslabel}{}{\hspace{.5em}\titlerule*[4bp]{$\cdot$}\contentspage}
\titlecontents{section}[2em]{\normalsize\songti\large}{\thecontentslabel}{}{\hspace{.5em}\titlerule*[4bp]{$\cdot$}\contentspage}
%---------------------------------------------------------------------
% 文档开始
%---------------------------------------------------------------------
\begin{document}
%---------------------------------------------------------------------
% 封面
%---------------------------------------------------------------------
\begin{titlepage}
    \begin{center}
        \includegraphics[width=0.7\textwidth]{figure/zhongcai.png}\\
        \vspace{8mm}
        \textbf{\fontsize{36bp}{17bp}\songti{本科生毕业论文(设计)}}\\\vspace{89bp}
        \textbf{\fontsize{22bp}{17bp} {\setmainfont[AutoFakeBold=2.17, AutoFakeSlant]{SimHei} \heiti  \MYTITLE}}\\ % 论文大标题为2号
        \vspace{103bp}
        {\songti\LARGE
            \begin{tabular}{rp{6.2cm}<{\centering}}
                {\makebox[4\ccwd][s]{学生姓名:}}            & \underline{\makebox[6cm]{\MYNAME}}    \\[9bp]
                {\makebox[4\ccwd][s]{学\hspace{2em} 号:}} & \underline{\makebox[6cm]{\MYID}}      \\[9bp]
                {\makebox[4\ccwd][s]{学\hspace{2em} 院:}} & \underline{\makebox[6cm]{\MYSCHOOL}}  \\[9bp]
                {\makebox[4\ccwd][s]{专\hspace{2em} 业:}} & \underline{\makebox[6cm]{\MYMAJOR}}   \\[9bp]
                {\makebox[4\ccwd][s]{指导教师:}}            & \underline{\makebox[6cm]{\MYADVISOR}} \\[9bp]
                {\makebox[4\ccwd][s]{日\hspace{2em} 期:}} & \underline{\makebox[6cm]{\MYDATE}}    \\[9bp]
            \end{tabular}
        }\\
    \end{center}
\end{titlepage}

%---------------------------------------------------------------------
% 摘要页
%---------------------------------------------------------------------
\setcounter{page}{1}
\thispagestyle{plain}
\begin{cnabstract}

    此处是摘要示例,请在abstract.tex中编辑摘要。

    119季秋天,本应是满月的夜晚,月亮却有一点点瑕疵。人类或许难以察觉,但妖怪们却对此十分敏感。

    为了夺回幻想乡的满月,妖怪们各自拉上熟识的人类,博丽灵梦和八云紫、雾雨魔理沙和爱丽丝·玛格特洛依德、十六夜咲夜和蕾米莉亚·斯卡蕾特以及魂魄妖梦和西行寺幽幽子两两一组,停止了夜晚,并踏上了解决异变的道路。

    沿途击败了莉格露·奈特巴格和米斯蒂娅·萝蕾拉后,自机们发现上白泽慧音为了保护人类,用能力将人类村落隐藏了起来。

    战斗过后,自机们在慧音的指引下进入了迷途竹林,并在其中遭遇了同样来调查异变的其他人类主人公。

    战胜对方后,自机们进入永远亭,打败了因幡天为和守护着走廊的铃仙·优昙华院·因幡,最终见到了异变的始作俑者——八意永琳和莱山辉夜。

    从月球逃亡到地上的辉夜,担心满月成为地月之间的通道,进而引来追兵,便命令永琳制造了幻影。虚假之月切断了通道,使地上成为巨大的密室,却也影响了幻想乡中的妖怪。
    辉夜败北后,得知幻想乡有结界保护,月亮上的追兵本就无法到达,便归还了真实之月。
    \par\textbf{关键字:}关键字1 \hspace{1em} 关键字2 \hspace{1em} 关键字3
\end{cnabstract}

\newpage
\setcounter{page}{1}
\thispagestyle{plain}
\begin{enabstract}

    This paper employs A-share listed companies in Shanghai and Shenzhen as samples to explore the impact of corporate innovation on corporate performance……
    
    \textbf{KEY WORDS: }corporate innovation\hspace{1em}corporate erformance\hspace{1em}resource-based view
\end{enabstract}

\newpage
%---------------------------------------------------------------------
% 目录页
%---------------------------------------------------------------------
\thispagestyle{plain}
\setcounter{page}{1}
\tableofcontents % 生成目录
\newpage
%---------------------------------------------------------------------
% 大标题
%---------------------------------------------------------------------
\begin{center}
    \textbf{\Large\MYTITLE}
    \par \vspace{-10bp}
\end{center}
\setcounter{page}{1}
%---------------------------------------------------------------------
% 正文
%---------------------------------------------------------------------
\chapter{一级标题示例}
此处开始正文,分别对应chapter1.tex、chapter2.tex、chapter3.tex

论文引用示例\cite{王宣承-1},文献按照国标2015格式引用。

脚注\footnote{脚注实例,每一页会重新标号}引用示例\footnote{脚注实例,每一页会重新标号}。

一些需要注意的问题:
\begin{enumerate}
    \item 英文字体全部采用Times New Roman
    \item 表格字号请注意设置为small
    \item 如希望在图表题注上注明脚注,请参考示例
\end{enumerate}

\section{穢き世の美しき檻}

\begin{table}[H]
    \centering
    \small
    \caption{东方幻想乡六名角色的分工}
    \begin{tabularx}{\textwidth}{X >{\centering\arraybackslash}X}
        \toprule[1.0bp]
        成员    & 分工      \\
        \midrule[0.75bp]
        博丽灵梦  & 乐园的可爱巫女 \\
        雾雨魔理沙 & 普通的魔法使  \\
        东风谷早苗 & 祭祀风的人类  \\
        十六夜咲夜 & 十六夜宵夜   \\
        魂魄妖梦  & 十六夜宵夜   \\
        \bottomrule[1.0bp]
    \end{tabularx}
    \vspace{4bp}
\end{table}

咲夜拥有拥有操纵时间的能力。

操纵时间的能力是最高等级的强大能力[东方求闻史纪],并不是一般人能通过修行取得的。
咲夜自称,她的时间停止能力,其实为超高速的移动能力[茨歌仙第35话]。
通过自如地运用这份能力,咲夜可以做到瞬间移动自己和物体、在时间停止中扫除、让竹子立刻开花[红魔乡魔理沙线]、拓宽红魔馆的内部空间[幻想揭示板]等等,当然也可用于战斗。

咲夜还是使用飞刀的高手,至于其精准度,如果让妖精女仆头顶苹果站在十二丈外,据说她投出的小刀可以正中妖精女仆的额头。[东方求闻史纪]
咲夜拥有高超的厨艺。高超的厨艺与红魔馆的豪华食材相搭配(特别是肉类食材),使她在宴会上有着很高的人气。[茨歌仙第33话]
咲夜是一位表面上的魔术师,因为她的魔术不需要手法和技术,只是单纯地操纵了时间[幻想揭示板]。 说白了就是在时间暂停期间移动物品的位置等等,在常人眼中看起来就像是凭空变出来的一样。

\chapter{丑时一刻}
\section{二级标题示例}
\subsection{三级标题示例}
拆行公式:
\begin{equation}
    \begin{split}
        UNEMSEC = \beta_0 + \beta_1HEA\_0 + \beta_2HEA\_1 + \beta_3OLD\_0 + \\
        \beta_4OLD\_1 + \beta_5ifiwork + \beta_6family\_income + \epsilon
    \end{split}
\end{equation}

\lipsum[1]

\section{玛格特罗伊德}
脚注示例\footnote{って、こりゃまた随分集まったわね。}

「穢き所に、いかでか久しくおはせん。」

そういうと閉ざされた扉は一枚残らず開き――

引用实例,注意该格式未在文件中规定:
\begin{quotation}

    永琳、私の力でもう一度だけチャンスをあげる。

    これで負けたらその時は……。

    そこの人妖!

    私の力で作られた薬と永琳の本当の力、
    一生忘れないものになるよ!
\end{quotation}

私は輝夜。

\section{线性回归计算peincome、unincome}
\subsection{被解释变量的选择}
关于这两个变量,原文的描述是:
\lipsum[2]

北风卷第白草折,

胡天八月即飞雪

交叉引用示例:表~\ref{hhh}~
\begin{equation*}
    Ave\_income = \beta_0 + \beta_{1}Ave\_age + \beta_{2}Ave\_edu + \beta_{3}hgender + \beta_{4}hccp + \beta_{5}worker\_ratio + \epsilon
\end{equation*}


\subsection{解释变量的选择}
これで永夜の術は破れて、夜は明ける!
\begin{table}[H]
    \small
    \caption{手动插入表格示例}
    \centering
    \begin{tabularx}{\textwidth}{X >{\centering\arraybackslash}X >{\centering\arraybackslash}X >{\centering\arraybackslash}X >{\centering\arraybackslash}X}
        \toprule[1.0bp]
        variable & mean & sd   & min   & max   \\
        \midrule[0.75bp]
        SR1      & 0.60 & 0.52 & -5.00 & 1.00  \\
        SR2      & 0.47 & 0.63 & -5.38 & 1.00  \\
        peincome & 9.72 & 0.60 & 7.86  & 11.92 \\
        unincome & 0.00 & 0.74 & -3.35 & 3.71  \\
        PENSION  & 0.78 & 0.42 & 0.00  & 1.00  \\
        HEASEC   & 0.93 & 0.26 & 0.00  & 1.00  \\
        UNEMSEC  & 0.45 & 0.50 & 0.00  & 1.00  \\
        r        & 0.61 & 0.27 & 0.00  & 1.00  \\
        pension  & 0.47 & 0.34 & 0.00  & 1.00  \\
        heasec   & 0.57 & 0.30 & 0.00  & 1.00  \\
        unemsec  & 0.29 & 0.35 & 0.00  & 1.00  \\
        \bottomrule[1.0bp]
    \end{tabularx}
    \label{hhh}
    \vspace{4bp}
\end{table}

\chapter{正文章节}

\section{二级标题示例}

插入图片示例:
\begin{figure}[H]
    \small
    \centering
    \vspace{22bp}
    \includegraphics[width=0.65\textwidth]{figure/cancha.pdf}
    \caption[Caption for LOF]{插入图片实例\protect\footnotemark}
\end{figure}
\footnotetext{注意,如果希望在题注上标注数据来源,需要按照该示例写脚注}

\section{二级标题示例}

后续可以继续添加chapter4.tex、chapter5.tex以添加更多章节,注意需要在main.tex中使用input命令加入。

%---------------------------------------------------------------------
% 参考文献
%---------------------------------------------------------------------
\specialchapter
\phantomsection
\addcontentsline{toc}{chapter}{参考文献}
\printbibliography
\nocite{*} %显示所有文献
\newpage
%---------------------------------------------------------------------
% 其他
%---------------------------------------------------------------------
\chapter{中央财经大学本科毕业论文(设计)原创性声明}

本人郑重声明:所提交的毕业论文(设计)《\MYTITLE》,是本人在指导老师的指导下独立进行研究工作所取得的成果。除文中已经注明引用的内容外,不含任何其他个人或集体已经发表或撰写过的作品成果,不存在购买、由他人代写、剽窃和伪造数据等作假行为。对本文研究/设计做出重要贡献的个人和集体,均已在文中以明确方式标明。本人完全意识到本声明的法律结果,如违反有关规定或上述声明,愿意承担由此产生的一切后果。

\vspace{44bp}
\noindent 论文作者签名:\hspace*{18em}签字日期:\hspace*{2em}年\hspace*{1em}月\hspace*{1em}日

\vspace{66bp}

\chapter*{本科毕业论文(设计)版权使用授权书}

本人完全了解中央财经大学有权保留并向国家有关部门或机构送交本论文的复印件和磁盘,允许论文被查阅和借阅。本人授权中央财经大学可以将本人的毕业论文(设计)的全部或部分内容编入有关数据库进行检索和传播,可以采用影印、缩印或扫描等复制手段保存、汇编论文。

\vspace{44bp}
\noindent 论文作者签名:\hspace*{18em}导师签名:

\vspace{22bp}
\noindent 签字日期:\hspace*{2em}年\hspace*{1em}月\hspace*{1em}日\hspace*{13em}签字日期:\hspace*{2em}年\hspace*{1em}月\hspace*{1em}日

\newpage

\chapter{致\hspace{0.5em}谢}

致谢内容

\vspace{66pt}

\hfill 张三

\hfill 2024年4月10日

\newpage
%---------------------------------------------------------------------
% 附录
%---------------------------------------------------------------------
插入代码示例:

\begin{lstlisting}[style=python,title={Python code}] 
#PythonDraw.py
import turtle as t
t.setup(650, 350, 200, 200)
t.penup()
t.fd(-250)
t.pendown()
t.pensize(25)
t.pencolor("purple color")
t.seth(-40)
for i in range(4):
    t.circle(40, 80)
    t.circle(-40, 80)
    t.circle(40, 80/2)
    t.fd(40)
    t.circle(16, 180)
    t.fd(40 * 2/3)
    t.done()
\end{lstlisting}



\end{document}
