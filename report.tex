%---------------------------------------------------------------------
%	个性化信息
%--------------------------------------------------------------------
\newcommand{\MYTITLE}{污秽之世,美丽之笼}  %论文标题
\newcommand{\MYID}{2018300000}  %学号
\newcommand{\MYNAME}{藤原妹红}  %姓名
\newcommand{\MYSCHOOL}{金融学院}  %学院
\newcommand{\MYMAJOR}{博丽神社}  %专业
\newcommand{\MYADVISOR}{上白泽慧音}  %指导教师
\newcommand{\MYDATE}{2022年1月7日}  %日期
%---------------------------------------------------------------------
%   各种导言
%--------------------------------------------------------------------
\documentclass[a4paper,12pt]{report}

\usepackage{geometry} % to change the page dimensions
\geometry{a4paper,left=2.5cm,right=2.5cm,top=2.54cm,bottom=2.54cm}%页边距
\usepackage{ctex}
\usepackage{xeCJK}
\usepackage{comment}
\usepackage{setspace}
\usepackage{fancyhdr}
\usepackage{graphicx}
\usepackage{wrapfig}
\usepackage{subfigure}
\usepackage{array}
\usepackage{titlesec}
\usepackage{titletoc}
\usepackage[titletoc]{appendix}
%\usepackage[top=30mm,bottom=30mm,left=20mm,right=20mm]{geometry}
%\usepackage{cite}

%\usepackage{courier}
\setmonofont{Courier New}
\usepackage{listings}

\usepackage{scrextend}
%\usepackage{perpage}
%\MakePerPage[1]{footnote}
\usepackage[perpage]{footmisc}
\deffootnote[1em]{1em}{1em}{\zihao{5}\thefootnotemark.\space}  %脚注
%---------------------------------------------------------------------
%   参考文献设置
%--------------------------------------------------------------------
\usepackage[backend = biber, style = references/gb7714-2015, defernumbers=true]{biblatex}
\renewcommand*{\bibfont}{\small}
\addbibresource{references/bibtest.bib}
\renewcommand{\bibname}{参考文献}
%---------------------------------------------------------------------
%	引用文献设置为上标
%---------------------------------------------------------------------
\begin{comment}
    \makeatletter
    \def\@cite#1#2{\textsuperscript{[{#1\if@tempswa , #2\fi}]}}
    \makeatother
\end{comment}

\lstset{tabsize=4, keepspaces=true,
    xleftmargin=2em,xrightmargin=0em, aboveskip=0.1em,
    %backgroundcolor=\color{gray!20},  % 定义背景颜色
    frame=none,                       % 表示不要边框
    extendedchars=false,              % 解决代码跨页时,章节标题,页眉等汉字不显示的问题
    numberstyle=\ttfamily,
    basicstyle=\ttfamily,
    keywordstyle=\color{blue}\bfseries,
    breakindent=10pt,
    identifierstyle=,                 % nothing happens
    commentstyle=\color{green}\small,  % 注释的设置
    morecomment=[l][\color{green}]{\#},
    numbers=left,stepnumber=1,numberstyle=\scriptsize,
    showstringspaces=false,
    showspaces=false,
    flexiblecolumns=true,
    breaklines=true, breakautoindent=true,breakindent=4em,
    escapeinside={/*@}{@*/},
}
\usepackage{amsmath}
\usepackage{amsthm}
\newtheorem{theorem}{定理}
\newtheorem{definition}{定义}
\newtheorem{corollary}{推论}
\newtheorem{example}{例}
\renewcommand {\thetable} {\arabic{table}}
\renewcommand {\thefigure} {\arabic{figure}}
\usepackage{amsfonts}
\usepackage{lipsum}
%\usepackage{bm}
\usepackage{booktabs} % for much better looking tables
\usepackage{paralist} % very flexible & customisable lists (eg. enumerate/itemize, etc.)
\usepackage{verbatim} % adds environment for commenting out blocks of text & for better verbatim
\usepackage{subfigure} % make it possible to include more than one captioned figure/table in a single float
% These packages are all incorporated in the memoir class to one degree or another...
\usepackage{cases} %equation set
\usepackage{multirow} %use table
\usepackage{algorithm}
\usepackage{algorithmic}
%\usepackage{cite}
\usepackage{hyperref}
\usepackage{longtable}
\usepackage[font={small}]{caption}  % 这样设置表头字体好像是10.95号
\usepackage{zhnumber} % change section number to chinese
\hypersetup{colorlinks,linkcolor=black,anchorcolor=black,citecolor=black, pdfstartview=FitH,bookmarksnumbered=true,bookmarksopen=true,} % set href in tex & pdf
%\usepackage[framed,numbered,autolinebreaks,useliterate]{mcode} % 插入matlab代码
\XeTeXlinebreaklocale "zh"
\XeTeXlinebreakskip = 0pt plus 1pt minus 0.1pt
\setlength{\baselineskip}{22pt}
%---------------------------------------------------------------------
%	图表顺序标号,不分章节
%---------------------------------------------------------------------
\usepackage{chngcntr}
\usepackage[OT1]{fontenc} 
\counterwithout{table}{chapter}
\counterwithout{table}{section}
\counterwithout{figure}{chapter}
\counterwithout{figure}{section}
\counterwithout{equation}{chapter}
\counterwithout{equation}{section}
\titleclass{\chapter}{straight}%禁止chapter换页
%---------------------------------------------------------------------
%	页眉页脚设置
%---------------------------------------------------------------------
\pagestyle{fancy}
\fancyhead[C]{\MYTITLE}
\lhead{}
\rhead{}
\cfoot{\thepage}
\newcommand{\myfont}{\fontfamily{ptm}\fontseries{b}\selectfont}
% \newcommand*{\mysfont}{\fontfamily{timesbd}\selectfont}
%---------------------------------------------------------------------
%	标题格式设置
%---------------------------------------------------------------------
\renewcommand\thesection{(\zhnum{section})}
\renewcommand \thesubsection {\arabic{subsection}.}
\renewcommand \thechapter {\zhnum{chapter}、}
\titleformat{\chapter}{\centering\zihao{4}\songti\bfseries}{\chinese{chapter}、}{0.00em}{}
\titlespacing{\chapter}{12pt}{12pt}{*3}  %空行直接按照12pt算的,即小四号
\titlespacing{\section}{0pt}{0pt}{*0}
\titlespacing{\subsection}{0pt}{0pt}{*0}
\titleformat{\section}{\zihao{-4}\songti\bfseries}{$\qquad$(\chinese{section})}{0.05em}{}
\titleformat{\subsection}{\zihao{-4}\songti\mdseries}{$\qquad$\arabic{subsection}.$\ $}{0.05em}{}
\renewcommand{\figurename}{图}
\renewcommand{\tablename}{表}
%---------------------------------------------------------------------
%	摘要设置
%---------------------------------------------------------------------
%\renewcommand{\abstractname}{摘要}
\newcommand{\enabstractname}{ABSTRACT}
\newcommand{\cnabstractname}{内\quad 容\quad 摘\quad 要}
\newenvironment{enabstract}{%
  \par\small
  \noindent\mbox{}\hfill{{\zihao{3} \fontfamily{Times New Roman Bold}\selectfont 
  \textbf{\enabstractname}%
  }}\hfill\mbox{}\par
  \vskip 1.0ex}{\par\vskip 1.0ex}
\newenvironment{cnabstract}{%
  \par\small
  \noindent\mbox{}\hfill{\bfseries \zihao{3} \cnabstractname}\hfill\mbox{}\par
  \vskip 1.0ex}{\par\vskip 1.0ex}

%---------------------------------------------------------------------
%	目录页设置
%---------------------------------------------------------------------
%\renewcommand{\contentsname}{\zihao{-3} 目\quad 录}
\setcounter{tocdepth}{1}
\renewcommand{\contentsname}{\zihao{3}\bfseries\centering{目$\quad$录}}
\titlecontents{chapter}[0em]{\songti\zihao{4}\bfseries}{\thecontentslabel\ }{}
{\hspace{.5em}\titlerule*[4pt]{$\cdot$}\contentspage}
\titlecontents{section}[2em]{\vspace{0.1\baselineskip}\songti\zihao{4}}{\thecontentslabel\ }{}
{\hspace{.5em}\titlerule*[4pt]{$\cdot$}\contentspage}
%\titlecontents{subsection}[4em]{\vspace{0.1\baselineskip}\songti\zihao{-4}}{\thecontentslabel\ }{}
%{\hspace{.5em}\titlerule*[4pt]{$\cdot$}\contentspage}
\setCJKmainfont[AutoFakeBold = {2.17}]{SimSun}  % 全局中文字体设置

\begin{document}
% \setCJKfamilyfont{zhsong}{}  % 采用标准宋体 
% \renewcommand*{\songti}{\CJKfamily{zhsong}}
% \setCJKfamilyfont{zhsong}{宋体}
\setCJKfamilyfont{songti}[AutoFakeBold = {2.17}]{SimSun}
\renewcommand*{\songti}{\CJKfamily{songti}}
% \setmainfont{Times New Roman}
\setmainfont[AutoFakeBold = {2.17}]{SimSun}

% \setCJKmainfont{\songti}
%---------------------------------------------------------------------
%	封面设置
%---------------------------------------------------------------------
\begin{titlepage}
    \begin{center}
        \includegraphics[width=0.7\textwidth]{figure/zhongcai.png}\\
        \vspace{8mm}
        \textbf{\zihao{-0}\songti{本科生毕业论文(设计)}}\\
        \vspace{25mm}
        \textbf{\zihao{2}{\heiti\textbf{\MYTITLE}}}\\[0.8cm]
        \vspace{50mm}
        % \vspace{\fill}
        % \setlength{\extrarowheight}{3mm}
        {\songti\zihao{3}
            \begin{tabular}{rp{8.2cm}<{\centering}}
                % \specialrule{0em}{30pt}
                {\makebox[4\ccwd][s]{学生姓名:}}    & \underline{\makebox[8cm]{\MYNAME}}     \\[10pt]
                {\makebox[4\ccwd][s]{学\qquad 号:}}    & \underline{\makebox[8cm]{\MYID}}       \\[10pt]
                {\makebox[4\ccwd][s]{学\qquad 院:}}    & \underline{\makebox[8cm]{\MYSCHOOL}}     \\[10pt]
                {\makebox[4\ccwd][s]{专\qquad 业:}}    & \underline{\makebox[8cm]{\MYMAJOR}}    \\[10pt]
                {\makebox[4\ccwd][s]{指导教师:}}       & \underline{\makebox[8cm]{\MYADVISOR}}  \\[10pt]
                {\makebox[4\ccwd][s]{日\qquad 期:}}    & \underline{\makebox[8cm]{\MYDATE}}            \\[10pt]
            \end{tabular}
        }\\[2cm]
    \end{center}
\end{titlepage}
%---------------------------------------------------------------------
%  摘要页
%---------------------------------------------------------------------
\setcounter{page}{1}
\thispagestyle{plain}
\begin{cnabstract}
    \vspace{12pt}
    此处是摘要示例,请在abstract.tex中编辑摘要。

    119季秋天,本应是满月的夜晚,月亮却有一点点瑕疵。人类或许难以察觉,但妖怪们却对此十分敏感。
    为了夺回幻想乡的满月,妖怪们各自拉上熟识的人类,博丽灵梦和八云紫、雾雨魔理沙和爱丽丝·玛格特洛依德、十六夜咲夜和蕾米莉亚·斯卡蕾特以及魂魄妖梦和西行寺幽幽子两两一组,停止了夜晚,并踏上了解决异变的道路。
    沿途击败了莉格露·奈特巴格和米斯蒂娅·萝蕾拉后,自机们发现上白泽慧音为了保护人类,用能力将人类村落隐藏了起来。
    战斗过后,自机们在慧音的指引下进入了迷途竹林,并在其中遭遇了同样来调查异变的其他人类主人公。
    战胜对方后,自机们进入永远亭,打败了因幡天为和守护着走廊的铃仙·优昙华院·因幡,最终见到了异变的始作俑者——八意永琳和蓬莱山辉夜。
    从月球逃亡到地上的辉夜,担心满月成为地月之间的通道,进而引来追兵,便命令永琳制造了幻影。虚假之月切断了通道,使地上成为巨大的密室,却也影响了幻想乡中的妖怪。
    辉夜败北后,得知幻想乡有结界保护,月亮上的追兵本就无法到达,便归还了真实之月。
    \par\textbf{关键字:}关键字1 \qquad 关键字2 \qquad 关键字3
\end{cnabstract}
\vspace{12pt}
\setmainfont{Times New Roman}
\begin{enabstract}
    % \begin{myfont}
        \vspace{12pt}
        \lipsum[1]
        \par\textbf{KEY WORDS:} keyword1 \qquad keyword2 \qquad keyword3
    % \end{myfont}
\end{enabstract}
\setmainfont[AutoFakeBold = {2.17}]{宋体}  %  将全局西文字体也都设置为宋体


\newpage

%---------------------------------------------------------------------
%  目录页
%---------------------------------------------------------------------
\thispagestyle{plain}
\setcounter{page}{1}
\begin{spacing}{1.375} % TODO: 这个行距有问题
\tableofcontents % 生成目录
\end{spacing}
\newpage
%---------------------------------------------------------------------
%  引言
%---------------------------------------------------------------------
\begin{center}
    \textbf{\zihao{-3}{\MYTITLE}}
    \\[12pt]
\end{center}
\thispagestyle{plain}
\setcounter{page}{1}

本作包含四组自机,每组自机由人类(高速)与妖怪(低速)组成。全组合通关后,可以使用单人/单妖机体进行游戏(对话、ed与组合的对话相同),共计十二个机体。
对于第四面,boss会根据自机而变化;第五面通关时,可以自行选择进入6A面或6B面(二者剧情不同,通关后分别为normal ending和good ending)。
本作主要系统为与剧情相关的刻符系统,以吃到的蓝点道具数量获得残机,决死bomb进行了一定改动,添加了(针对单张的)符卡练习模式。
%---------------------------------------------------------------------
%  正文
%---------------------------------------------------------------------
\chapter{引言}

\section{研究背景}

此处开始正文,分别对应chapter1.tex、chapter2.tex、chapter3.tex

论文引用示例\cite{王宣承-1},文献按照国标2015格式引用。

脚注\footnote{脚注实例,每一页会重新标号}引用示例\footnote{脚注实例,每一页会重新标号}。

一些需要注意的问题:
\begin{enumerate}
    \item 英文字体全部采用Times New Roman
    \item 表格字号请注意设置为small
    \item 如希望在图表题注上注明脚注,请参考示例
\end{enumerate}

\section{研究目的}

表格示例:
\begin{table}[H]
    \centering
    \small
    \caption{东方幻想乡六名角色的分工}
    \begin{tabularx}{\textwidth}{X >{\centering\arraybackslash}X}
        \toprule[1.0bp]
        成员    & 分工      \\
        \midrule[0.75bp]
        博丽灵梦  & 乐园的可爱巫女 \\
        雾雨魔理沙 & 普通的魔法使  \\
        东风谷早苗 & 祭祀风的人类  \\
        十六夜咲夜 & 十六夜宵夜   \\
        魂魄妖梦  & 十六夜宵夜   \\
        \bottomrule[1.0bp]
    \end{tabularx}
    \vspace{4bp}
\end{table}

\section{研究意义}

\section{研究思路}

\section{研究方法}

\chapter{文献综述}
\section{二级标题示例}
\subsection{三级标题示例}
拆行公式:
\begin{equation}
    \begin{split}
        UNEMSEC = \beta_0 + \beta_1HEA\_0 + \beta_2HEA\_1 + \beta_3OLD\_0 + \\
        \beta_4OLD\_1 + \beta_5ifiwork + \beta_6family\_income + \epsilon
    \end{split}
\end{equation}

\section{二级标题示例}

\subsection{被解释变量的选择}

不编号公式示例:
\[Ave\_income = \beta_0 + \beta_{1}Ave\_age + \beta_{2}Ave\_edu + \beta_{3}hgender + \beta_{4}hccp + \beta_{5}worker\_ratio + \epsilon\]


\subsection{解释变量的选择}

交叉引用示例:表~\ref{hhh}~
\begin{table}[H]
    \small
    \caption{手动插入表格示例}
    \centering
    \begin{tabularx}{\textwidth}{X >{\centering\arraybackslash}X >{\centering\arraybackslash}X >{\centering\arraybackslash}X >{\centering\arraybackslash}X}
        \toprule[1.0bp]
        variable & mean & sd   & min   & max   \\
        \midrule[0.75bp]
        SR1      & 0.60 & 0.52 & -5.00 & 1.00  \\
        SR2      & 0.47 & 0.63 & -5.38 & 1.00  \\
        peincome & 9.72 & 0.60 & 7.86  & 11.92 \\
        unincome & 0.00 & 0.74 & -3.35 & 3.71  \\
        PENSION  & 0.78 & 0.42 & 0.00  & 1.00  \\
        HEASEC   & 0.93 & 0.26 & 0.00  & 1.00  \\
        UNEMSEC  & 0.45 & 0.50 & 0.00  & 1.00  \\
        r        & 0.61 & 0.27 & 0.00  & 1.00  \\
        pension  & 0.47 & 0.34 & 0.00  & 1.00  \\
        heasec   & 0.57 & 0.30 & 0.00  & 1.00  \\
        unemsec  & 0.29 & 0.35 & 0.00  & 1.00  \\
        \bottomrule[1.0bp]
    \end{tabularx}
    \label{hhh}
    \vspace{4bp}
\end{table}

若表格后面紧接着一级标题,需要手动使用vspace命令控制行距。

\chapter{正文章节}

\section{二级标题示例}

插入图片示例:
\begin{figure}[H]
    \small
    \centering
    \vspace{22bp}
    \includegraphics[width=0.65\textwidth]{figure/cancha.pdf}
    \caption[Caption for LOF]{插入图片实例\protect\footnotemark}
\end{figure}
\footnotetext{注意,如果希望在题注上标注数据来源,需要按照该示例写脚注}

\section{二级标题示例}

后续可以继续添加chapter4.tex、chapter5.tex以添加更多章节,注意需要在main.tex中使用input命令加入。

%---------------------------------------------------------------------
%  参考文献
%---------------------------------------------------------------------

\printbibliography
\newpage
\chapter*{中央财经大学本科毕业论文(设计)原创性声明}
\addcontentsline{toc}{chapter}{中央财经大学本科毕业论文(设计)原创性声明}
\vspace{1mm}
本人郑重声明:所提交的毕业论文(设计)《\MYTITLE》,
是本人在指导老师的指导下独立进行研究工作所取得的成果。
除文中已经注明引用的内容外,不含任何其他个人或集体已经发表或撰写过的作品成果,
不存在购买、由他人代写、剽窃和伪造数据等作假行为。
对本文研究/设计做出重要贡献的个人和集体,均已在文中以明确方式标明。
本人完全意识到本声明的法律结果,如违反有关规定或上述声明,愿意承担由此产生的一切后果。
\\[50pt]
作者签名:\\
\rightline{年 \quad 月 \quad 日}

\newpage
\chapter*{致谢}
\addcontentsline{toc}{chapter}{致谢}
それはともかく、そんなぬるいゲームの二次創作はどうかというと、こ
れがまた熱い(笑)。先に言ったとおりこのゲームの特徴は、弾幕がキャ
ラクターとストーリーを語る所。この部分を意識しているのかしていない
のか判りませんが、頂いたどの作品もちゃんとキャラクターが強くて楽し
まさせていただきました。キャラクターの強さは弾幕の強烈さと同義です。
(頂いた作品は全て見させて頂いた後、全て保管してあります。既に数百
作品にものぼるという。)個々の作品に対しての感想は私の立場上、公の
場で余り言う事が出来なくなってしまいましたが、それは個々に挨拶する
時に……。
\newpage
\chapter*{附录}
\addcontentsline{toc}{chapter}{附录}

\end{document}
