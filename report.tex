%---------------------------------------------------------------------
%	 个人信息
%---------------------------------------------------------------------
\newcommand{\MYTITLE}{污秽之世,美丽之笼}  %论文标题
\newcommand{\MYID}{2018300000}  %学号
\newcommand{\MYNAME}{藤原妹红}  %姓名
\newcommand{\MYSCHOOL}{金融学院}  %学院
\newcommand{\MYMAJOR}{博丽神社}  %专业
\newcommand{\MYADVISOR}{上白泽慧音}  %指导教师
\newcommand{\MYDATE}{2022年1月7日}  %日期
%---------------------------------------------------------------------
%    文档类型
%---------------------------------------------------------------------
\documentclass[a4paper, 12bp]{report}
%---------------------------------------------------------------------
%    导入宏包
%---------------------------------------------------------------------
\usepackage{geometry} % 改变页面尺寸
\usepackage{ctex} % 中文支持
\usepackage{xeCJK} % 支持CJK字体,适用于中文、日文和韩文
\usepackage{anyfontsize} % 支持任意字号
\usepackage{comment} % 添加注释环境
\usepackage{setspace} % 设置行间距
\usepackage{fancyhdr} % 自定义页眉页脚
\usepackage{graphicx} % 插入图像
\usepackage{wrapfig} % 文字环绕图像
\usepackage{subcaption} % 替代过时的 subfigure,支持子图与子标题
\usepackage{float} % 浮动体控制
\usepackage{array} % 增强表格功能
\usepackage{titlesec} % 修改章节标题样式
\usepackage{titletoc} % 修改目录样式
\usepackage[titletoc]{appendix} % 附录设置
\usepackage{listings, color} % 插入代码
\usepackage{xcolor} % 代码高亮
\usepackage{algpseudocode} % 代替过时的 algorithmic,用于编写伪代码
\usepackage{algorithm} % 插入算法伪代码环境
\usepackage{scrextend} % KOMA-Script 扩展
\usepackage[perpage]{footmisc} % 脚注相关设置
\usepackage{pifont} % 提供圆圈数字等特殊字符
\usepackage[OT1]{fontenc} % 字体编码为 OT1
\usepackage{inputenc} % 输入编码为 utf8
\usepackage{fontspec} % 西文字体设置
\usepackage[backend=biber, style=gb7714-2015, defernumbers=true]{biblatex} % 参考文献管理
\usepackage{amsmath} % 数学公式
\usepackage{amsthm} % 定理环境
\newtheorem{theorem}{定理}
\newtheorem{definition}{定义}
\newtheorem{corollary}{推论}
\newtheorem{example}{例}
\renewcommand {\thetable} {\arabic{table}} % 表格编号为阿拉伯数字
\renewcommand {\thefigure} {\arabic{figure}} % 图形编号为阿拉伯数字
\usepackage{amsfonts} % 数学字体
\usepackage{lipsum} % 生成随机文本
\usepackage{longtable, booktabs} % 更优美的表格
\usepackage{paralist} % 更灵活的列表(customisable lists)
\usepackage{verbatim} % 更好的 verbatim 环境
\usepackage{cases} % 公式组环境
\usepackage{multirow} % 在表格中跨多行合并单元格
\usepackage{tabularx, booktabs} % 提供额外的表格功能
\usepackage{zhnumber} % 章节编号为中文
\usepackage{hyperref} % 超链接支持
\usepackage[font=small]{caption} % 设置图表标题为五号
\usepackage{chngcntr} % 允许改变计数器的前缀
%---------------------------------------------------------------------
%     页边距设置
%---------------------------------------------------------------------
\geometry{a4paper,left=2.5cm,right=2.5cm,top=2.54cm,bottom=2.54cm}
%---------------------------------------------------------------------
%     脚注设置
%---------------------------------------------------------------------
\renewcommand\thefootnote{\ding{\numexpr171+\value{footnote}}}
% 脚注字体设置为small
\renewcommand{\footnotesize}{\small}
\deffootnote[1em]{1em}{1em}{\small\thefootnotemark\space}
% 脚注的横线
\renewcommand{\footnoterule}{
    \kern -3bp
    \hrule width 0.25\paperwidth height 1bp
    \kern 2bp
}
%---------------------------------------------------------------------
%     参考文献设置
%---------------------------------------------------------------------
\renewcommand*{\bibfont}{\normalsize}
\addbibresource{references/bibtest.bib}
\renewcommand{\bibname}{参考文献}
%---------------------------------------------------------------------
%	引用文献设置为上标
%---------------------------------------------------------------------
\begin{comment}
        \makeatletter
        \def\@cite#1#2{\textsuperscript{[{#1\if@tempswa , #2\fi}]}}
        \makeatother
\end{comment}
%---------------------------------------------------------------------
%     附录代码设置
%---------------------------------------------------------------------
% Python highlights color settings
\definecolor{pBasic}{RGB}{248, 248, 242}      % 默认
\definecolor{pKeyword}{RGB}{228,0,128}        % 洋红色
\definecolor{pString}{RGB}{148,0,209}         % 紫色
\definecolor{pComment}{RGB}{117,113,94}       % 灰色
\definecolor{pIdentifier}{RGB}{166, 226, 46}  % 绿色
\definecolor{pBackground}{RGB}{245,245,245}   % 浅灰色
\definecolor{pNumber}{RGB}{128,128,128}       % 灰色
\lstdefinestyle{python}{
    language=python,               % 语言
    xleftmargin=25pt,
    xrightmargin=15pt,
    frame=tlbr,framesep=4pt,framerule=0pt, % 边框设置
    % frame=shadowbox,rulesepcolor=\color{red!20!green!20!blue!20}, % 边框设置
    basicstyle=\normalsize\ttfamily,          % 代码字体、大小
    basicstyle=\footnotesize\fontspec{Consolas},
    keywordstyle=\color{pKeyword},       % 关键字颜色
    stringstyle=\color{pString},         % 字符串颜色
    commentstyle=\color{pComment},       % 注释颜色
    backgroundcolor=\color{pBackground}, % 背景颜色
    emph={format_string,eff_ana_bf,permute,eff_ana_btr}, % 自定义函数
    emphstyle=\color{pIdentifier},       % 自定义函数颜色
    showspaces=false,                    % 显示空格
    showstringspaces=false,              % 用下划线显示空格
    showtabs=false,                      % 显示tab
    tabsize=4,                           % 设置默认缩进空格数
    captionpos=t,                        % 设置caption在顶部
    breaklines=true,                     % 自动换行
    numberstyle=\tiny\color{pNumber},
    numbers=left,                        % 行号位置
    stepnumber=1,                        % 行号计数间隔
    %numbersep=5pt,                      % 行号与代码间距
}
%---------------------------------------------------------------------
%    超链接设置
%---------------------------------------------------------------------
\hypersetup{colorlinks,linkcolor=black,anchorcolor=black,citecolor=black, pdfstartview=FitH,bookmarksnumbered=true,bookmarksopen=true,} % 设置引用样式
\XeTeXlinebreaklocale "zh"
\XeTeXlinebreakskip = 0bp plus 1bp minus 0.1bp % 微调 XeTeX 的行间距
%---------------------------------------------------------------------
%    列表间距设置
%---------------------------------------------------------------------
\let\enumerate\compactenum
\let\endenumerate\endcompactenum
\let\itemize\compactitem
\let\enditemize\endcompactitem
\let\description\compactdesc
\let\enddescription\endcompactdesc
%---------------------------------------------------------------------
%    图表上下行间距设置
%---------------------------------------------------------------------
\captionsetup[table]{position=above,belowskip=17bp}
\captionsetup[figure]{position=below,belowskip=17bp}
%---------------------------------------------------------------------
%    图表顺序标号,不分章节
%---------------------------------------------------------------------
\counterwithout{table}{chapter}
\counterwithout{table}{section}
\counterwithout{figure}{chapter}
\counterwithout{figure}{section}
\counterwithout{equation}{chapter}
\counterwithout{equation}{section}
\titleclass{\chapter}{straight}%禁止chapter换页
%---------------------------------------------------------------------
%    页眉页脚设置
%---------------------------------------------------------------------
\pagestyle{fancy}
\fancyhead[C]{\small\MYTITLE}
\lhead{}
\rhead{}
% 页眉去除横线
% \renewcommand{\headrulewidth}{0bp}
% \cfoot{\thepage \\ \textcolor{red}{请注意格式问题可能会导致拒绝答辩。以任何形式采用该模板意味着您已承认:使用该模板而引发的一切负面或正面后果与任何你以外的人都没有任何关系。}}

%---------------------------------------------------------------------
%    调整字体
%---------------------------------------------------------------------
\setCJKmainfont{simsun.ttc}[AutoFakeBold=2.17]
\setCJKsansfont{simhei.ttf}[AutoFakeBold=2.17]
\setCJKmonofont{simfang.ttf}
\setCJKfamilyfont{zhsong}{simsun.ttc}[AutoFakeBold=2.17]
\setCJKfamilyfont{zhhei}{simhei.ttf}[AutoFakeBold=2.17]
\setCJKfamilyfont{zhkai}{simkai.ttf}
\setCJKfamilyfont{zhfs}{simfang.ttf}
\renewcommand*{\songti}{\CJKfamily{zhsong}}
\renewcommand*{\heiti}{\CJKfamily{zhhei}}
\renewcommand*{\kaishu}{\CJKfamily{zhkai}}
\renewcommand*{\fangsong}{\CJKfamily{zhfs}}
\makeatletter
% 重定义\normalsize(小四)来设置字体为12bp和行间距为22bp
\renewcommand\normalsize{%
    \@setfontsize\normalsize{12bp}{17bp}%
    \abovedisplayskip 17\p@ \@plus0\p@ \@minus0\p@
    \abovedisplayshortskip \z@ \@plus0\p@
    \belowdisplayshortskip 17\p@ \@plus0\p@ \@minus0\p@
    \belowdisplayskip \abovedisplayskip
    \let\@listi\@listI}
% 重定义\small(五号)来设置字体为10.5bp和行间距为22bp
\renewcommand\small{%
    \@setfontsize\small{10.5bp}{17bp}%
    \abovedisplayskip 17\p@ \@plus0\p@ \@minus0\p@
    \abovedisplayshortskip \z@ \@plus0\p@
    \belowdisplayshortskip 17\p@ \@plus0\p@ \@minus0\p@
    \belowdisplayskip \abovedisplayskip
    \let\@listi\@listI}
% 重定义\large(四号)来设置字体为14bp和行间距为22bp
\renewcommand\large{%
    \@setfontsize\large{14bp}{17bp}%
    \abovedisplayskip 17\p@ \@plus0\p@ \@minus0\p@
    \abovedisplayshortskip \z@ \@plus0\p@
    \belowdisplayshortskip 17\p@ \@plus0\p@ \@minus0\p@
    \belowdisplayskip \abovedisplayskip
    \let\@listi\@listI}
% 重定义\Large(小三)来设置字体为15bp和行间距为22bp
\renewcommand\Large{%
    \@setfontsize\Large{15bp}{17bp}%
    \abovedisplayskip 17\p@ \@plus0\p@ \@minus0\p@
    \abovedisplayshortskip \z@ \@plus0\p@
    \belowdisplayshortskip 17\p@ \@plus0\p@ \@minus0\p@
    \belowdisplayskip \abovedisplayskip
    \let\@listi\@listI}
% 重定义\LARGE(三号)来设置字体为16bp和行间距为22bp
\renewcommand\LARGE{%
    \@setfontsize\LARGE{16bp}{17bp}%
    \abovedisplayskip 17\p@ \@plus0\p@ \@minus0\p@
    \abovedisplayshortskip \z@ \@plus0\p@
    \belowdisplayshortskip 17\p@ \@plus0\p@ \@minus0\p@
    \belowdisplayskip \abovedisplayskip
    \let\@listi\@listI}
\makeatother
%---------------------------------------------------------------------
%    标题格式设置
%---------------------------------------------------------------------
\renewcommand\thesection{(\zhnum{section})}
\renewcommand \thesubsection {\arabic{subsection}.}
\renewcommand \thechapter {\zhnum{chapter}、}
\titleformat{\chapter}{\centering\large\songti\bfseries}{\chinese{chapter}、}{0em}{}
\titleformat{\section}{\normalsize\songti\bfseries}{\hspace{2em}(\chinese{section})}{0em}{}
\titleformat{\subsection}{\normalsize\songti\mdseries}{\hspace{2em}\arabic{subsection}.$\ $}{0em}{}

\titlespacing{\chapter}{0bp}{22bp}{22bp} % 一级标题前后隔一行小四
\titlespacing{\section}{0bp}{0bp}{*0}
\titlespacing{\subsection}{0bp}{0bp}{*0}

\renewcommand{\figurename}{图}
\renewcommand{\tablename}{表}
%---------------------------------------------------------------------
%    摘要设置
%---------------------------------------------------------------------
\newcommand{\enabstractname}{ABSTRACT}
\newcommand{\cnabstractname}{内\quad 容\quad 摘\quad 要}
\newenvironment{cnabstract}{%
    \begin{center}
        \songti \bfseries \LARGE \cnabstractname \vspace{12bp}
    \end{center}
    \setlength{\parindent}{2em}
}{%
    \par \vspace{12bp}
}

\newenvironment{enabstract}{%
    \begin{center}
        \selectfont \bfseries \LARGE \enabstractname \vspace{10bp}
    \end{center}
    \setlength{\parindent}{2em}
}{%
    \par % \vspace{22bp}
}
%---------------------------------------------------------------------
%    目录页设置
%---------------------------------------------------------------------
\setcounter{tocdepth}{1}
\renewcommand{\contentsname}{\LARGE\bfseries\centering{目\hspace{2em}录}}
\titlecontents{chapter}[0em]{\songti\large\bfseries}{\thecontentslabel}{}
{\hspace{.5em}\titlerule*[4bp]{$\cdot$}\contentspage}
\titlecontents{section}[2em]{\songti\large}{\thecontentslabel}{}
{\hspace{.5em}\titlerule*[4bp]{$\cdot$}\contentspage}
%\titlecontents{subsection}[4em]{\vspace{0.1\baselineskip}\songti\normalsize}{\thecontentslabel\ }{}
%{\hspace{.5em}\titlerule*[4bp]{$\cdot$}\contentspage}

%---------------------------------------------------------------------
%    文档开始
%---------------------------------------------------------------------
\begin{document}
%---------------------------------------------------------------------
%    封面
%---------------------------------------------------------------------
\begin{titlepage}
    \begin{center}
        \includegraphics[width=0.7\textwidth]{figure/zhongcai.png}\\
        \vspace{8mm}
        \textbf{\fontsize{36bp}{17bp}\songti{本科生毕业论文(设计)}}\\\vspace{89bp}
        \textbf{\fontsize{22bp}{17bp} {\heiti \textbf{\MYTITLE}}}\\ % 论文大标题为2号
        \vspace{103bp}
        {\songti\LARGE
            \begin{tabular}{rp{6.2cm}<{\centering}}
                {\makebox[4\ccwd][s]{学生姓名:}}            & \underline{\makebox[6cm]{\MYNAME}}    \\[9bp]
                {\makebox[4\ccwd][s]{学\hspace{2em} 号:}} & \underline{\makebox[6cm]{\MYID}}      \\[9bp]
                {\makebox[4\ccwd][s]{学\hspace{2em} 院:}} & \underline{\makebox[6cm]{\MYSCHOOL}}  \\[9bp]
                {\makebox[4\ccwd][s]{专\hspace{2em} 业:}} & \underline{\makebox[6cm]{\MYMAJOR}}   \\[9bp]
                {\makebox[4\ccwd][s]{指导教师:}}            & \underline{\makebox[6cm]{\MYADVISOR}} \\[9bp]
                {\makebox[4\ccwd][s]{日\hspace{2em} 期:}} & \underline{\makebox[6cm]{\MYDATE}}    \\[9bp]
            \end{tabular}
        }\\
    \end{center}
\end{titlepage}
%---------------------------------------------------------------------
%    摘要页
%---------------------------------------------------------------------
\setcounter{page}{1}
\thispagestyle{plain}
\begin{cnabstract}
    \vspace{12pt}
    此处是摘要示例,请在abstract.tex中编辑摘要。

    119季秋天,本应是满月的夜晚,月亮却有一点点瑕疵。人类或许难以察觉,但妖怪们却对此十分敏感。
    为了夺回幻想乡的满月,妖怪们各自拉上熟识的人类,博丽灵梦和八云紫、雾雨魔理沙和爱丽丝·玛格特洛依德、十六夜咲夜和蕾米莉亚·斯卡蕾特以及魂魄妖梦和西行寺幽幽子两两一组,停止了夜晚,并踏上了解决异变的道路。
    沿途击败了莉格露·奈特巴格和米斯蒂娅·萝蕾拉后,自机们发现上白泽慧音为了保护人类,用能力将人类村落隐藏了起来。
    战斗过后,自机们在慧音的指引下进入了迷途竹林,并在其中遭遇了同样来调查异变的其他人类主人公。
    战胜对方后,自机们进入永远亭,打败了因幡天为和守护着走廊的铃仙·优昙华院·因幡,最终见到了异变的始作俑者——八意永琳和蓬莱山辉夜。
    从月球逃亡到地上的辉夜,担心满月成为地月之间的通道,进而引来追兵,便命令永琳制造了幻影。虚假之月切断了通道,使地上成为巨大的密室,却也影响了幻想乡中的妖怪。
    辉夜败北后,得知幻想乡有结界保护,月亮上的追兵本就无法到达,便归还了真实之月。
    \par\textbf{关键字:}关键字1 \qquad 关键字2 \qquad 关键字3
\end{cnabstract}
\vspace{12pt}
\setmainfont{Times New Roman}
\begin{enabstract}
    % \begin{myfont}
        \vspace{12pt}
        \lipsum[1]
        \par\textbf{KEY WORDS:} keyword1 \qquad keyword2 \qquad keyword3
    % \end{myfont}
\end{enabstract}
\setmainfont[AutoFakeBold = {2.17}]{宋体}  %  将全局西文字体也都设置为宋体


\newpage

%---------------------------------------------------------------------
%    目录页
%---------------------------------------------------------------------
\thispagestyle{plain}
\setcounter{page}{1}
\tableofcontents % 生成目录
\newpage
%---------------------------------------------------------------------
%    引言
%---------------------------------------------------------------------
\begin{center}
    \textbf{\Large{\MYTITLE}}
    \\[22bp]
\end{center}
\setcounter{page}{1}

本作包含四组自机,每组自机由人类(高速)与妖怪(低速)组成。全组合通关后,可以使用单人/单妖机体进行游戏(对话、ed与组合的对话相同),共计十二个机体。
对于第四面,boss会根据自机而变化;第五面通关时,可以自行选择进入6A面或6B面(二者剧情不同,通关后分别为normal ending和good ending)。
本作主要系统为与剧情相关的刻符系统,以吃到的蓝点道具数量获得残机,决死bomb进行了一定改动,添加了(针对单张的)符卡练习模式。
%---------------------------------------------------------------------
%    正文
%---------------------------------------------------------------------
\chapter{引言}

\section{研究背景}

此处开始正文,分别对应chapter1.tex、chapter2.tex、chapter3.tex

论文引用示例\cite{王宣承-1},文献按照国标2015格式引用。

脚注\footnote{脚注实例,每一页会重新标号}引用示例\footnote{脚注实例,每一页会重新标号}。

一些需要注意的问题:
\begin{enumerate}
    \item 英文字体全部采用Times New Roman
    \item 表格字号请注意设置为small
    \item 如希望在图表题注上注明脚注,请参考示例
\end{enumerate}

\section{研究目的}

表格示例:
\begin{table}[H]
    \centering
    \small
    \caption{东方幻想乡六名角色的分工}
    \begin{tabularx}{\textwidth}{X >{\centering\arraybackslash}X}
        \toprule[1.0bp]
        成员    & 分工      \\
        \midrule[0.75bp]
        博丽灵梦  & 乐园的可爱巫女 \\
        雾雨魔理沙 & 普通的魔法使  \\
        东风谷早苗 & 祭祀风的人类  \\
        十六夜咲夜 & 十六夜宵夜   \\
        魂魄妖梦  & 十六夜宵夜   \\
        \bottomrule[1.0bp]
    \end{tabularx}
    \vspace{4bp}
\end{table}

\section{研究意义}

\section{研究思路}

\section{研究方法}

\chapter{文献综述}
\section{二级标题示例}
\subsection{三级标题示例}
拆行公式:
\begin{equation}
    \begin{split}
        UNEMSEC = \beta_0 + \beta_1HEA\_0 + \beta_2HEA\_1 + \beta_3OLD\_0 + \\
        \beta_4OLD\_1 + \beta_5ifiwork + \beta_6family\_income + \epsilon
    \end{split}
\end{equation}

\section{二级标题示例}

\subsection{被解释变量的选择}

不编号公式示例:
\[Ave\_income = \beta_0 + \beta_{1}Ave\_age + \beta_{2}Ave\_edu + \beta_{3}hgender + \beta_{4}hccp + \beta_{5}worker\_ratio + \epsilon\]


\subsection{解释变量的选择}

交叉引用示例:表~\ref{hhh}~
\begin{table}[H]
    \small
    \caption{手动插入表格示例}
    \centering
    \begin{tabularx}{\textwidth}{X >{\centering\arraybackslash}X >{\centering\arraybackslash}X >{\centering\arraybackslash}X >{\centering\arraybackslash}X}
        \toprule[1.0bp]
        variable & mean & sd   & min   & max   \\
        \midrule[0.75bp]
        SR1      & 0.60 & 0.52 & -5.00 & 1.00  \\
        SR2      & 0.47 & 0.63 & -5.38 & 1.00  \\
        peincome & 9.72 & 0.60 & 7.86  & 11.92 \\
        unincome & 0.00 & 0.74 & -3.35 & 3.71  \\
        PENSION  & 0.78 & 0.42 & 0.00  & 1.00  \\
        HEASEC   & 0.93 & 0.26 & 0.00  & 1.00  \\
        UNEMSEC  & 0.45 & 0.50 & 0.00  & 1.00  \\
        r        & 0.61 & 0.27 & 0.00  & 1.00  \\
        pension  & 0.47 & 0.34 & 0.00  & 1.00  \\
        heasec   & 0.57 & 0.30 & 0.00  & 1.00  \\
        unemsec  & 0.29 & 0.35 & 0.00  & 1.00  \\
        \bottomrule[1.0bp]
    \end{tabularx}
    \label{hhh}
    \vspace{4bp}
\end{table}

若表格后面紧接着一级标题,需要手动使用vspace命令控制行距。

\chapter{正文章节}

\section{二级标题示例}

插入图片示例:
\begin{figure}[H]
    \small
    \centering
    \vspace{22bp}
    \includegraphics[width=0.65\textwidth]{figure/cancha.pdf}
    \caption[Caption for LOF]{插入图片实例\protect\footnotemark}
\end{figure}
\footnotetext{注意,如果希望在题注上标注数据来源,需要按照该示例写脚注}

\section{二级标题示例}

后续可以继续添加chapter4.tex、chapter5.tex以添加更多章节,注意需要在main.tex中使用input命令加入。

%---------------------------------------------------------------------
%    其他
%---------------------------------------------------------------------
\printbibliography
\addcontentsline{toc}{chapter}{参考文献}
\newpage
\chapter*{中央财经大学本科毕业论文(设计)原创性声明}
\addcontentsline{toc}{chapter}{中央财经大学本科毕业论文(设计)原创性声明}
本人郑重声明:所提交的毕业论文(设计)《\MYTITLE》,是本人在指导老师的指导下独立进行研究工作所取得的成果。除文中已经注明引用的内容外,不含任何其他个人或集体已经发表或撰写过的作品成果,不存在购买、由他人代写、剽窃和伪造数据等作假行为。对本文研究/设计做出重要贡献的个人和集体,均已在文中以明确方式标明。本人完全意识到本声明的法律结果,如违反有关规定或上述声明,愿意承担由此产生的一切后果。

\vspace{66bp}
作者签名:\hfill 2024年4月10日

\newpage
\chapter*{致谢}

\addcontentsline{toc}{chapter}{致谢}
それはともかく、そんなぬるいゲームの二次創作はどうかというと、こ
れがまた熱い(笑)。先に言ったとおりこのゲームの特徴は、弾幕がキャ
ラクターとストーリーを語る所。この部分を意識しているのかしていない
のか判りませんが、頂いたどの作品もちゃんとキャラクターが強くて楽し
まさせていただきました。キャラクターの強さは弾幕の強烈さと同義です。
(頂いた作品は全て見させて頂いた後、全て保管してあります。既に数百
作品にものぼるという。)個々の作品に対しての感想は私の立場上、公の
場で余り言う事が出来なくなってしまいましたが、それは個々に挨拶する
時に……。
\newpage
\chapter*{附录}

\addcontentsline{toc}{chapter}{附录}
插入代码示例:
\begin{lstlisting}[language=C]
    qui reg SR1 $xx dummy1-dummy24 if time==0
    predict e1,res
    g e2 = e1^2
    g lne2 = log(e2)
    qui reg lne2 peincome if time==0,noc
    predict lne2f
    g e2f =exp(lne2f)
    reg SR1 $xx dummy1-dummy24 if time==0 [aw=1/e2f]
\end{lstlisting}

\end{document}
