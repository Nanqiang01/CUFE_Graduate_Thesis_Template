\chapter{丑时一刻}
\section{二级标题示例}
\subsection{三级标题示例}
拆行公式:
\begin{equation}
    \begin{split}
        UNEMSEC = \beta_0 + \beta_1HEA\_0 + \beta_2HEA\_1 + \beta_3OLD\_0 + \\
        \beta_4OLD\_1 + \beta_5ifiwork + \beta_6family\_income + \epsilon
    \end{split}
\end{equation}

\lipsum[1]

\section{玛格特罗伊德}
脚注示例\footnote{って、こりゃまた随分集まったわね。}

「穢き所に、いかでか久しくおはせん。」

そういうと閉ざされた扉は一枚残らず開き――

引用实例,注意该格式未在文件中规定:
\begin{quotation}

    永琳、私の力でもう一度だけチャンスをあげる。

    これで負けたらその時は……。

    そこの人妖!

    私の力で作られた薬と永琳の本当の力、
    一生忘れないものになるよ!
\end{quotation}

私は輝夜。

\section{线性回归计算peincome、unincome}
\subsection{被解释变量的选择}
关于这两个变量,原文的描述是:
\lipsum[2]

北风卷第白草折,

胡天八月即飞雪

交叉引用示例:表~\ref{hhh}~
\begin{equation*}
    Ave\_income = \beta_0 + \beta_{1}Ave\_age + \beta_{2}Ave\_edu + \beta_{3}hgender + \beta_{4}hccp + \beta_{5}worker\_ratio + \epsilon
\end{equation*}


\subsection{解释变量的选择}
これで永夜の術は破れて、夜は明ける!
\begin{table}[H]
    \small
    \caption{手动插入表格示例}
    \centering
    \begin{tabularx}{\textwidth}{X >{\centering\arraybackslash}X >{\centering\arraybackslash}X >{\centering\arraybackslash}X >{\centering\arraybackslash}X}
        \toprule[1.0bp]
        variable & mean & sd   & min   & max   \\
        \midrule[0.75bp]
        SR1      & 0.60 & 0.52 & -5.00 & 1.00  \\
        SR2      & 0.47 & 0.63 & -5.38 & 1.00  \\
        peincome & 9.72 & 0.60 & 7.86  & 11.92 \\
        unincome & 0.00 & 0.74 & -3.35 & 3.71  \\
        PENSION  & 0.78 & 0.42 & 0.00  & 1.00  \\
        HEASEC   & 0.93 & 0.26 & 0.00  & 1.00  \\
        UNEMSEC  & 0.45 & 0.50 & 0.00  & 1.00  \\
        r        & 0.61 & 0.27 & 0.00  & 1.00  \\
        pension  & 0.47 & 0.34 & 0.00  & 1.00  \\
        heasec   & 0.57 & 0.30 & 0.00  & 1.00  \\
        unemsec  & 0.29 & 0.35 & 0.00  & 1.00  \\
        \bottomrule[1.0bp]
    \end{tabularx}
    \label{hhh}
    \vspace{4bp}
\end{table}
